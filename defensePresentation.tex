% $Header: /Users/joseph/Documents/LaTeX/beamer/solutions/conference-talks/conference-ornate-20min.en.tex,v 90e850259b8b 2007/01/28 20:48:30 tantau $

\documentclass{beamer}
\usepackage{hyperref}
\usefonttheme{serif} % default family is serif
%\usepackage{fontspec}
%\setmainfont{Times New Roman}
% This file is a solution template for:

% - Talk at a conference/colloquium.
% - Talk length is about 20min.
% - Style is ornate.

% Copyright 2004 by Till Tantau <tantau@users.sourceforge.net>.
%
% In principle, this file can be redistributed and/or modified under
% the terms of the GNU Public License, version 2.
%
% However, this file is supposed to be a template to be modified
% for your own needs. For this reason, if you use this file as a
% template and not specifically distribute it as part of a another
% package/program, I grant the extra permission to freely copy and
% modify this file as you see fit and even to delete this copyright
% notice. 

\usepackage{pdfpages}
\mode<presentation>
{
  \usetheme{Warsaw}
  % or ...
  \setbeamercovered{transparent}
  % or whatever (possibly just delete it)
}
\usepackage[english]{babel}
% or whatever


\usepackage{times}
\usepackage[T1]{fontenc}
% Or whatever. Note that the encoding and the font should match. If T1
% does not look nice, try deleting the line with the fontenc.


\title[] % (optional, use only with long paper titles)
{Compositional and Analytic Applications of Automated Music Notation via Object-oriented Programming:}

\subtitle
{Compositional Applications}

\author[Trevi\~{n}o, Jeffrey] % (optional, use only with lots of authors)
{Jeffrey Trevi\~{n}o\inst{1}}
% - Give the names in the same order as the appear in the paper.
% - Use the \inst{?} command only if the authors have different
%   affiliation.

\institute[University of California, San Diego] % (optional, but mostly needed)
{
  \inst{1}%
  Department of Music\\
  University of California, San Diego}
% - Use the \inst command only if there are several affiliations.
% - Keep it simple, no one is interested in your street address.

\date[CFP 2003] % (optional, should be abbreviation of conference name)
{Doctoral Dissertation Defense (June 3, 2013)}
% - Either use conference name or its abbreviation.
% - Not really informative to the audience, more for people (including
%   yourself) who are reading the slides online

\subject{Music}
% This is only inserted into the PDF information catalog. Can be left
% out. 



% If you have a file called "university-logo-filename.xxx", where xxx
% is a graphic format that can be processed by latex or pdflatex,
% resp., then you can add a logo as follows:

 \pgfdeclareimage[height=1.5cm,width=1.5cm]{university-logo}{university-logo}
 \logo{\pgfuseimage{university-logo}}



% Delete this, if you do not want the table of contents to pop up at
% the beginning of each subsection:
%\AtBeginSubsection[]
%{
  %\begin{frame}<beamer>{Outline}
    %\tableofcontents[currentsection,currentsubsection]
  %\end{frame}
%}


% If you wish to uncover everything in a step-wise fashion, uncomment
% the following command: 

%\beamerdefaultoverlayspecification{<+->}


\begin{document}

\begin{frame}
  \titlepage
\end{frame}

\begin{frame}{Introduction}
  \tableofcontents
  % You might wish to add the option [pausesections]
\end{frame}


% Structuring a talk is a difficult task and the following structure
% may not be suitable. Here are some rules that apply for this
% solution: 

% - Exactly two or three sections (other than the summary).
% - At *most* three subsections per section.
% - Talk about 30s to 2min per frame. So there should be between about
%   15 and 30 frames, all told.

% - A conference audience is likely to know very little of what you
%   are going to talk about. So *simplify*!
% - In a 20min talk, getting the main ideas across is hard
%   enough. Leave out details, even if it means being less precise than
%   you think necessary.
% - If you omit details that are vital to the proof/implementation,
%   just say so once. Everybody will be happy with that.

\section{Algorithmic Tendencies, 2004-2008}

\subsection{Mobiles}
\begin{frame}{Mobiles --- Graphic Expressions of Indeterminate Systems}
  \begin{itemize}
  \item
    \emph{\href{binaryExperiment}{Binary Experiment for James Tenney}} (2005)
    \pause
    
   \item
     \emph{\href{mobileForSax}{Mobile for Tenor Saxophone}} (2005)
    \pause
   \item
    \emph{\href{sodaParaphrase}{Mexican Apple Soda Paraphrase}} (2007)
  \end{itemize}
\end{frame}

\subsection{Extended Common Notation}

\begin{frame}{Extended Common Notation}
  \begin{itemize}
  \item
    Formal Conceits
    \begin{itemize}
    \item<2->
    \emph{\href{sjmcForm}{Substitute Judgment}} (2004)
    \item<3->
    \emph{\href{zoetropes}{Zoetropes}} (2005)
    \item<4->
    \emph{Mexican Apple Soda} (2006)
   \end{itemize}
  \item<5->
    Circumstance-specific Notations
    \begin{itemize}
    \item<6->
    \emph{\href{unitA}{Unit for Convenience and Better Living 003}} (2006)
    \item<7->
    \emph{Forty-two Statcoulombs} (2007)
    \item<8->
    \emph{\href{pfmemory}{Perfection Factory}} (2008)
  \end{itemize}
  \end{itemize}
\end{frame}

\section{Installation and Site-specific Work, 2009-2010}
\subsection{Fluxkonzerts/Hammer Museum Duos}
\begin{frame}{Fluxkonzerts/Hammer Museum Duos}
	\begin{itemize}
		\item{Fluxus at UCSD}
		\pause
		\item{10 concerts in formerly abandoned spaces, 2009-1010}
		\pause
		\item{\href{hammer}{4 Duos for the Hammer Museum, 2010}}
	\end{itemize}
		
\end{frame}

\subsection{Algorithmically Generated Trees}
\begin{frame}{Algorithmically Generated Trees}
	\begin{itemize}
		\item{\href{trees}{Cartoon Tree as Parametric Object}}
		\pause
		\item{Audience Participation}
	\end{itemize}
\end{frame}

\subsection{Blooms}
\begin{frame}{Blooms --- Looping Visual Music}
	\begin{itemize}
		\item{Visual Music at UCSD}
		\pause
		\item{{\href{bloom1}{Bloom 1}}}
		\pause
		\item{\href{bloom2}{Bloom 2}}
		\pause
		\item{Bloom 3}
	\end{itemize}
\end{frame}

\section{Computer-assisted Works, 2010-2013}
\subsection{Being Pollen}
\begin{frame}{Being Pollen}
\begin{itemize}
		\item{Rondo Alternating Between Solos and Duos.}
		\pause
		\item{\href{beingPollen}{Text-painting as relationship between Notation and Recording.}}
	\end{itemize}
\end{frame}

\subsection{+/-}
\begin{frame}{+/-}
\begin{itemize}
		\item{Representative of an experience.}
		\pause
		\item{Formal disposition.}
		\pause
		\item{\href{animation}{Flexibility of output.}}
	\end{itemize}
\end{frame}
\subsection{The World All Around}

\begin{frame}{The World All Around}

\begin{itemize}
		\item{Tribute.}
		\pause
		\item{Construction.}
		\pause
		\item{\href{worldFirst}{Refinement.}}
	\end{itemize}

\end{frame}

\section{Conclusion}

\subsection{Final Reflections}
\begin{frame}
\begin{itemize}
	\item{Continuity.}

		\begin{itemize}
		\item<2->
		Parametricism.
		\item<3->
		Change of Object Attributes Governs Development of Materials.
		\end{itemize}
	\item{Difference.}
		\begin{itemize}
		\item<4->
		Indeterminacy Relocated into Code.
		\item<5->
		Simpler Formal and Notational Ideas.
		\item<6->
		Less Contrast.
		\end{itemize}
	\item{Evaluation.}
		\begin{itemize}
		\item<7->
		Program as a Primary Evaluative Criterion.
		\item<8->
		Pragmatic Concerns.
		\end{itemize}
	
\end{itemize}
\end{frame}
\subsection{Toward Architecture}
\begin{frame}
	\begin{itemize}
		\item{\href{stage1}{Constrained randomness produces iterative structural constraint.}}
		\pause
		\item{\href{stage2}{Distributed, algorithmic stage deconstructs proscenium theater.}}
	\end{itemize}
\end{frame}
\end{document}


