\documentclass{sigplanconf}

\usepackage{amsmath}

\begin{document}

\special{papersize=8.5in,11in}
\setlength{\pdfpageheight}{\paperheight}
\setlength{\pdfpagewidth}{\paperwidth}

\conferenceinfo{CONF 'yy}{Month d--d, 20yy, City, ST, Country} 
\copyrightyear{20yy} 
\copyrightdata{978-1-nnnn-nnnn-n/yy/mm} 
\doi{nnnnnnn.nnnnnnn}

%\titlebanner{banner above paper title}        % These are ignored unless
%\preprintfooter{short description of paper}   % 'preprint' option specified.

\title{Generating LilyPond Code in Python via Abjad to Create Music Notation Documents}
%\subtitle{Subtitle Text, if any}

\authorinfo{Josiah Wolf Oberholtzer}
           {Harvard University}
           {josiah.oberholtzer@gmail.com}
\authorinfo{Jeffrey Trevi\~{n}o}
           {Colorado College}
           {jeffrey.trevino@gmail.com}
\authorinfo{Trevor Ba\v{c}a}
           {Harvard University}
           {trevor.baca@gmail.com}

\maketitle

\begin{abstract}
LilyPond is a modular, extensible, programmable music notation document
compiler that converts text input into automatically formatted graphic output.
The Abjad API for Formalized Score Control -- which extends the Python
programming language with an open-source, object-oriented model of
common-practice music notation that enables users to build music notation
documents through the aggregation of elemental music notation objects --
generates well-formed LilyPond input syntax. Abjad models hierarchically
organized music information as a reference-rich, normalized tree. In order to
model musical-typographical constructs such as slurs, beams and ties, Abjad
introduces cyclicity into its core tree data structure via bidirectionally
referenced, subtree-spanning nodes. Abjad traverses this data structure to
collect formatting information from each encountered node, and a hybrid set of
rules derived from graph theory, western music notation, and the constraints of
LilyPond's domain-specific language enables Abjad to organize this collected
information into a LilyPond input string.
\end{abstract}

\category{CR-number}{subcategory}{third-level}

\terms
term1, term2

\keywords
keyword1, keyword2

\section{Introduction}

The text of the paper begins here.

\appendix
\section{Appendix Title}

This is the text of the appendix, if you need one.

\acks

Acknowledgments, if needed.

% We recommend abbrvnat bibliography style.

\bibliographystyle{abbrvnat}

% The bibliography should be embedded for final submission.

\begin{thebibliography}{}
\softraggedright

\bibitem[Smith et~al.(2009)Smith, Jones]{smith02}
P. Q. Smith, and X. Y. Jones. ...reference text...

\end{thebibliography}


\end{document}