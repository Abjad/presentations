\section{About Abjad}

\begin{frame}[fragile]{Abjad history}
\begin{markdown}
- C into Finale via MIDI (1997)
- Mathematica into Sibelius via MIDI (2001)
- Mathematica into SCORE (2003)
- Mathematica into LilyPond (2004)
- Python into Adobe Illustrator (2004)
- Python into LilyPond (2005)
- Max/MSP into MS Access into Adobe Illustrator (2008)
- Public release on GoogleCode (2008)
- Migration to GitHub (2011)
- Abjad 2.16 released (2015)
\end{markdown}
\end{frame}

\begin{frame}[fragile]{Problems}
\begin{markdown}
We want:

- Ease of use
- Extensibility
- Stability
- Publication-quality typesetting
- To just make some music

We don't want:

- To learn how PostScript works
- To juggle Bezier curve control points
- To even think about kerning
\end{markdown}
\end{frame}

\begin{frame}{Stack}
\begin{table}
    \caption{Abjad's Software Stack}
    \begin{tabular}{ |c|c|c|c|c| }
        \hline
        \multicolumn{5}{|c|}{\textbf{Python}} \\
        \hline
        \multicolumn{5}{|c|}{\textbf{Abjad}} \\
        \hline
        \xcancel{\textbf{SCORE}} & \textbf{LilyPond} & \textbf{Steinberg?} & ... & ... \\
        \hline
    \end{tabular}
\end{table}
\end{frame}

\begin{frame}[fragile]{LaTeX, LilyPond \& Graphviz}
\begin{markdown}
- Automated typesetting systems
- What-You-See-Is-What-You-Mean
- Available on the command-line
- Extensible / modular / allow scripting
- Take plain-text as input
- Give back beautiful graphics as output 
- Oh, and Python isn't half-bad at...
    - Writing out plain text files, and...
    - Opening shells to command-line programs
\end{markdown}
\end{frame}