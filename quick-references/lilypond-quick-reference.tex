\documentclass[a4paper]{tufte-handout}

\newenvironment{abjadbookoutput}{}{}

\newcommand{\headerLeft}{\textsc{Abjad/CCRMA}}
\newcommand{\headerCenter}{\textsc{LilyPond Quick Reference}}
\newcommand{\headerRight}{\textsc{August $|$ 2017}}

\usepackage[export]{adjustbox}
\usepackage{csquotes}
\usepackage{multicol}
\usepackage{graphicx}
\usepackage{lab_notes}
\usepackage{fancyvrb}
\fvset{fontsize=\normalsize}
\usepackage{minted}
\usemintedstyle{bw}
\setminted{%
    autogobble=true,
    fontsize=\small,
}

\title{LilyPond Quick Reference}
\date{2017}

\begin{document}
\maketitle

\section{Invocation}

LilyPond is a command-line tool, and it must be run from inside your computer's
\emph{terminal}.

\begin{description}
    \item [Compile one file:]
        \hfill \verb|lilypond my-file.ly|
    \item [Compile multiple files:]
        \hfill \verb|lilypond one.ly two.ly three.ly|
    \item [Get help:]
        \hfill \verb|lilypond --help|
    \item [Print LilyPond's version:]
        \hfill \verb|lilypond --version|
    \item [Choose where the output goes:]
        \hfill \verb|lilypond -o out.pdf in.ly|
\end{description}

\hrulefill{}

\section{General Syntax}

\begin{description}
    \item [Functions and variables:]
        \footnote{When referencing a previously defined function or variable
        in LilyPond, preceed the name with a slash. This syntax style is taken
        from \LaTeX, a \enquote{what-you-see-is-what-you-mean} typesetting
        program that inspired the original LilyPond developers.}
        \hfill \verb|\xxxx|

    \item [Code blocks:]
        \hfill \verb|{ ... }|

    \item [Variable assignment:]
        \hfill \verb|var = { ... }|

    \item [Line comment:]
        \hfill \begin{minted}{text}
        % this is a line comment
        \end{minted}

    \item [Block comment:]
        \hfill \begin{minted}{text}
        %{
            this is a multi-line
            block comment
        %}
        \end{minted}

    \item [Post-events:]
        \footnote{\enquote{Post-events} include any commands following a note
        definition that are pertinent to that note, including articulations,
        markup, and starting or stopping \enquote{spanners} such as beams,
        slurs and glissandi.}
        \hfill \verb|c\foo|

    \item [Directions:]
        \hfill \verb|c^"up" c_"down" c-"center"|

    \item [Scheme lists, pairs, booleans and symbols:]\footnote{
        LilyPond embeds \emph{another} programming language, called
        \enquote{Scheme}, for advanced control over typography and typesetting
        behavior. Scheme is a variant of Lisp. LilyPond uses the \texttt{\#}
        character to indicate when the syntax is changing from LilyPond's
        native syntax into Scheme.}
        \hfill \begin{minted}{text}
        #'(1 2 3)  % a Scheme list of three numbers
        #'(X . Y)  % a Scheme pair of the variable X and Y
        ##t ##f    % booleans
        #'symbol   % a symbol
        \end{minted}

    \item[Includes:]
        \hfill \verb|\include "another-file.ly"|

    \item[Version:]\footnote{
        Every \emph{top-level} LilyPond file you compile should start with a
        version command, otherwise LilyPond will complain.}
        \hfill \verb|\version "2.19.0"|

\end{description}

\hrulefill{}

\pagebreak

\section{Common Notation}

\begin{description}
    \item[Input language:]
        \hfill \verb|\language "english"|

    \item[Note names:]\footnote{%
        LilyPond's original developers were from the Netherlands, so notes use
        the Dutch spelling by default. Abjad's developers studied music
        primarily in the United States, so we use the English spellings
        instead. LilyPond supports many international spellings.}
        \footnote{The octave indications are missing in the input for this example.}
        \hfill \vspace{-\baselineskip} \\
        \begin{multicols}{2}

            \begin{minted}{text}
            c d e f g a b
            \end{minted}

            \begin{comment}
            <lilypond>[stylesheet=stylesheet-minimal.ily]
            { c' d' e' f' g' a' b'}
            </lilypond>
            \end{comment}

            \begin{abjadbookoutput}
            \noindent\includegraphics[max width=\linewidth, right,]{assets/lilypond-795ec185ffbc4b4d5f499935543ef898.pdf}
            \end{abjadbookoutput}

        \end{multicols}

    \item[Alterations:]
        \hfill \vspace{-\baselineskip} \\
        \begin{multicols}{2}

            \begin{minted}{text}
            cs bf as css bff b b! b?
            \end{minted}

            \begin{comment}
            <lilypond>[stylesheet=stylesheet-minimal.ily]
            { cs' bf as css' bff b b! b? }
            </lilypond>
            \end{comment}

            \begin{abjadbookoutput}
            \noindent\includegraphics[max width=\linewidth, right,]{assets/lilypond-1b7ff09badb002a5112dd216810c39fa.pdf}
            \end{abjadbookoutput}

        \end{multicols}

    \item[Octaves:]\footnote{%
        LilyPond uses European octave tick notation.}
        \hfill \vspace{-\baselineskip} \\
        \begin{multicols}{2}

            \raggedcolumns
            \interlinepenalty=10000

            \begin{minted}{text}
            c,,, c,, c, c c' c'' c'''
            \end{minted}

            \vfill \null \columnbreak

            \begin{comment}
            <lilypond>[stylesheet=stylesheet-minimal.ily]
            {
                \clef "bass_15" c,,, c,,
                \clef "bass" c, c
                \clef "treble" c' c'' c'''
            }
            </lilypond>
            \end{comment}

            \begin{abjadbookoutput}
            \noindent\includegraphics[max width=\linewidth, right,]{assets/lilypond-a685e7de7e2c87ec41700cf7ee6d8032.pdf}
            \end{abjadbookoutput}

        \end{multicols}

    \item[Durations:]\footnote{%
        LilyPond allows users to omit durations when entering notes. The last
        entered duration will be used for the current note. Abjad \emph{always}
        prints the duration of every note.}
        \hfill \vspace{-0.5\baselineskip} \\

        \begin{minted}{text}
        g1 g2 g4 g8 g16 g4. g4.. g4...
        \end{minted}

        \begin{comment}
        <lilypond>stylesheet=stylesheet-minimal.ily]
        { g1 g2 g4 g8 g16 g4. g4.. g4... }
        </lilypond>
        \end{comment}

        \begin{abjadbookoutput}
        \noindent\includegraphics[max width=\linewidth, right,]{assets/lilypond-5807f7e3cf33e90264001261c9e1cd33.pdf}
        \end{abjadbookoutput}

    \item[Chords:]
        \hfill \vspace{-\baselineskip} \\
        \begin{multicols}{2}

            \begin{minted}{text}
            <c' e' g'>4 <f' c''>4 <d' df'' ds''>4
            \end{minted}

            \begin{comment}
            <lilypond>[stylesheet=stylesheet-minimal.ily]
            { <c' e' g'>4 <f' c''>4 <d' df'' ds''>4 }
            </lilypond>
            \end{comment}

            \begin{abjadbookoutput}
            \noindent\includegraphics[max width=\linewidth, right,]{assets/lilypond-3c02d0230f6bc52f81cf653cb71bb0ca.pdf}
            \end{abjadbookoutput}

        \end{multicols}

    \item[Rests and skips:]\footnote{%
        Typographic \emph{skips} take up musical time but do not have any
        graphic representation. They're typically used for positioning
        notational elements - e.g. intermittent voices - relative to each
        other.}
        \hfill \vspace{-\baselineskip} \\
        \begin{multicols}{2}

            \begin{minted}{text}
            r2 r4 r4.. r16 s4 r4
            \end{minted}

            \begin{comment}
            <lilypond>[stylesheet=stylesheet-minimal.ily]
            { r2 r4 r4.. r16 s4 r4 }
            </lilypond>
            \end{comment}

            \begin{abjadbookoutput}
            \noindent\includegraphics[max width=\linewidth, right,]{assets/lilypond-c4434b8378b7e4a45a4a6df1ba4a658f.pdf}
            \end{abjadbookoutput}

        \end{multicols}

    \item[Ties:]
        \hfill \vspace{-\baselineskip} \\
        \begin{multicols}{2}

            \begin{minted}{text}
            c'4 ~ c' c'2 ~ c'4 <c' a' f''> ~ c'
            \end{minted}

            \begin{comment}
            <lilypond>[stylesheet=stylesheet-minimal.ily]
            { c'4 ~ c' c'2 ~ c'4 <c' a' f''> ~ c' }
            </lilypond>
            \end{comment}

            \begin{abjadbookoutput}
            \noindent\includegraphics[max width=\linewidth, right,]{assets/lilypond-38362ff0acd48b8d15197392d9c67f2f.pdf}
            \end{abjadbookoutput}

        \end{multicols}

    \item[Beams:]\footnote{%
        LilyPond applies beams automatically unless you tell it otherwise.}
        \hfill \vspace{-\baselineskip} \\
        \begin{multicols}{2}

            \begin{minted}{text}
            g8 g8 g8 [ g g ] g8 [ g16 g16 ]
            \end{minted}

            \begin{comment}
            <lilypond>[stylesheet=stylesheet-minimal.ily]
            { g8 [ g8 ] g8 [ g g ] g8 [ g16 g16 ] }
            </lilypond>
            \end{comment}

            \begin{abjadbookoutput}
            \noindent\includegraphics[max width=\linewidth, right,]{assets/lilypond-c1e7ec324e5f97edd0d5595fde8c0770.pdf}
            \end{abjadbookoutput}

        \end{multicols}

    \item[Clefs:]
        \hfill \vspace{-\baselineskip} \\
        \begin{multicols}{2}

            \begin{minted}{text}
            \clef "treble" \clef "bass"
            \clef "tenor" \clef "treble_8"
            \end{minted}

            \begin{comment}
            <lilypond>[stylesheet=stylesheet-minimal.ily]
            {
                \clef "treble" c'4
                \clef "bass" c'4
                \clef "tenor" c'4
                \clef "treble_8" c'4
            }
            </lilypond>
            \end{comment}

            \begin{abjadbookoutput}
            \noindent\includegraphics[max width=\linewidth, right,]{assets/lilypond-aa6657f9e5267ee3ee338fca4e0ce481.pdf}
            \end{abjadbookoutput}

        \end{multicols}

    \item[Time signatures:]\footnote{%
        LilyPond doesn't model measures like other notation software. Measures
        are not containers, and notes, time signatures and bar lines are drawn
        independently of one another.}
        \hfill \vspace{-\baselineskip} \\
        \begin{multicols}{2}

            \begin{minted}{text}
            \time 4/4 \time 2/2
            \time 3/4 \time 3/8
            \time 7/10
            \end{minted}

            \begin{comment}
            <lilypond>
            \new Staff \with {
                \override BarLine.stencil = ##f
            } {
                \time 4/4 s4
                \time 2/2 s4
                \time 3/4 s4
                \time 3/8 s4
                \time 7/10 s4
            }
            </lilypond>
            \end{comment}

            \begin{abjadbookoutput}
            \noindent\includegraphics[max width=\linewidth, right,]{assets/lilypond-825b466a0dc7d15cc9c11f24cba7e722.pdf}
            \end{abjadbookoutput}

        \end{multicols}

    \item[Key signatures:]
        \hfill \vspace{-\baselineskip} \\
        \begin{multicols}{2}

            \begin{minted}{text}
            \key c \minor
            \key c \major
            \key b \dorian
            \end{minted}

            \begin{comment}
            <lilypond>[stylesheet=stylesheet-minimal.ily]
            {
                \key c \minor s4
                \key c \major s4
                \key b \dorian s4
            }
            </lilypond>
            \end{comment}

            \begin{abjadbookoutput}
            \noindent\includegraphics[max width=\linewidth, right,]{assets/lilypond-ffb3953922380af38c317c2b8d961ed8.pdf}
            \end{abjadbookoutput}

        \end{multicols}

    \item[Slurs and phrasing slurs:]
        \hfill \vspace{-\baselineskip} \\
        \begin{multicols}{2}

            \begin{minted}{text}
            c' ( a' ) c' \( b ( a ) g f \)
            \end{minted}

            \begin{comment}
            <lilypond>[stylesheet=stylesheet-minimal.ily]
            { c' ( a ) c' \( b ( a ) g f \) }
            </lilypond>
            \end{comment}

            \begin{abjadbookoutput}
            \noindent\includegraphics[max width=\linewidth, right,]{assets/lilypond-352de17d220fcaf4e3b634c9298604da.pdf}
            \end{abjadbookoutput}

        \end{multicols}

    \item[Dynamics, hairpins, etc.:]\footnote{%
        Crescendi and decrescendi will span over music until they encounter
        another dynamic event. Use \texttt{\textbackslash!} to terminate a
        dynamic line without the use of final dynamic.}
        \hfill \vspace{-\baselineskip} \\
        \begin{multicols}{2}

            \begin{minted}{text}
            g' \p \< a' b' c'' \f \> d'' e'' \!
            \end{minted}

            \begin{comment}
            <lilypond>[stylesheet=stylesheet-minimal.ily]
            { g' \p \< a' b' c'' \f \> d'' e'' \! }
            </lilypond>
            \end{comment}

            \begin{abjadbookoutput}
            \noindent\includegraphics[max width=\linewidth, right,]{assets/lilypond-a0bbf97c84883271e5d0197ffc052d37.pdf}
            \end{abjadbookoutput}

        \end{multicols}

    \item[Articulations and text markup:]
        \hfill \vspace{-\baselineskip} \\
        \begin{multicols}{2}

            \begin{minted}{text}
            c'4 -> c' -. c' ^| c' ^- _+
            c' \trill c' _\markup{ \flat }
            \end{minted}

            \begin{comment}
            <lilypond>[stylesheet=stylesheet-minimal.ily]
            {
                c'4 -> c' -. c' ^! c' ^- _+
                c' \trill c' _\markup{ \flat }
            }
            </lilypond>
            \end{comment}

            \begin{abjadbookoutput}
            \noindent\includegraphics[max width=\linewidth, right,]{assets/lilypond-d721485faa4be66ceab18c2576c91b92.pdf}
            \end{abjadbookoutput}

        \end{multicols}

    \item[Glissandi:]
        \hfill \vspace{-\baselineskip} \\
        \begin{multicols}{2}

            \begin{minted}{text}
            c' \glissando g' \glissando d'
            \end{minted}

            \begin{comment}
            <lilypond>[stylesheet=stylesheet-minimal.ily]
            { c' \glissando g' \glissando d' }
            </lilypond>
            \end{comment}

            \begin{abjadbookoutput}
            \noindent\includegraphics[max width=\linewidth, right,]{assets/lilypond-cdd403f63054a6908460d75688152887.pdf}
            \end{abjadbookoutput}

        \end{multicols}

    \item[Triplets and tuplets]
        \hfill \vspace{-\baselineskip} \\
        \begin{multicols}{2}

            \begin{minted}{text}
            \times 2/3 { c'4 d' e' }
            \times 4/5 { c' d' e' f' g' }
            \end{minted}

            \begin{comment}
            <lilypond>[stylesheet=stylesheet-minimal.ily]
            {
                \times 2/3 { c'4 d' e' }
                \times 4/5 { c' d' e' f' g' }
            }
            </lilypond>
            \end{comment}

            \begin{abjadbookoutput}
            \noindent\includegraphics[max width=\linewidth, right,]{assets/lilypond-6cedfc123da46112107e50e71783a040.pdf}
            \end{abjadbookoutput}

        \end{multicols}

    \item[Parallel music:]
        \hfill \vspace{-\baselineskip} \\
        \begin{multicols}{2}

            \begin{minted}{text}
            << { a'2 c''4 a' } \\ { f'4 g'4 f'2 } >>
            \end{minted}

            \begin{comment}
            <lilypond>[stylesheet=stylesheet-minimal.ily]
            << { a'2 c''4 a' } \\ { f'4 g'4 f'2 } >>
            </lilypond>
            \end{comment}

            \begin{abjadbookoutput}
            \noindent\includegraphics[max width=\linewidth, right,]{assets/lilypond-55e935e948e6083878fa8b7c74a48082.pdf}
            \end{abjadbookoutput}

        \end{multicols}

    \item[Metronome and rehearsal marks:]
        \hfill \vspace{-\baselineskip} \\
        \begin{multicols}{2}

            \begin{minted}{text}
            \tempo "Adagio" 4=60 c'2 c' c'
            \mark \default c' \mark "X"
            \end{minted}

            \begin{comment}
            <lilypond>
            {
                \tempo "Adagio" 4=60 c'2 c' c'
                \mark \default c' \mark "X"
            }
            </lilypond>
            \end{comment}

            \begin{abjadbookoutput}
            \noindent\includegraphics[max width=\linewidth, right,]{assets/lilypond-e2dbe9f0a0991e43b854088ad67b253c.pdf}
            \end{abjadbookoutput}

        \end{multicols}

    \pagebreak

    \item[Partial measures, bar line checks, bar lines:]
        \hfill \vspace{-\baselineskip} \\
        \begin{multicols}{2}

            \begin{minted}{text}
            \time 2/4
            \partial 8 c'8 | c'2 | c'
            \bar "||" c' \bar "|."
            \end{minted}
            \vfill\null \columnbreak

            \begin{comment}
            <lilypond>
            {
                \partial 8 c'8 | c'2 | c'
                \bar "||" c' \bar "|."
            }
            </lilypond>
            \end{comment}

            \begin{abjadbookoutput}
            \noindent\includegraphics[max width=\linewidth, right,]{assets/lilypond-c194b50e2d37809cf764705e7a4fe7df.pdf}
            \end{abjadbookoutput}

            \hfill
        \end{multicols}

    \item[Repeats:]
        \hfill \vspace{-0.5\baselineskip} \\

        \begin{minted}{text}
        \time 2/4
        c'2 \repeat volta 2 { d' }
        \alternative { { e' } { f' } } g'
        \end{minted}

        \begin{comment}
        <lilypond>
        {
            \time 2/4
            c'2 \repeat volta 2 { d' }
            \alternative { { e' } { f' } } g'
        }
        </lilypond>
        \end{comment}

        \begin{abjadbookoutput}
        \noindent\includegraphics[max width=\linewidth, right,]{assets/lilypond-e557790dad972e809923a1fb8d43457e.pdf}
        \end{abjadbookoutput}

    \item[Grace notes:]
        \hfill \vspace{-\baselineskip} \\
        \begin{multicols}{2}

            \begin{minted}{text}
            \grace d'8 c'4
            \appoggiatura d'8 c'4
            \acciaccatura d'8 c'4
            \end{minted}

            \begin{comment}
            <lilypond>
            {
                \grace d'8 c'4
                \appoggiatura d'8 c'4
                \acciaccatura d'8 c'4
            }
            </lilypond>
            \end{comment}

            \begin{abjadbookoutput}
            \noindent\includegraphics[max width=\linewidth, right,]{assets/lilypond-d9a98a62b89a1c0aa280bba45e3de8f4.pdf}
            \end{abjadbookoutput}

        \end{multicols}

    \item[Full/Multi-measure rests:]
        \hfill \vspace{-0.5\baselineskip} \\

        \begin{minted}{text}
        R1 | R1 * 3 | \time 3/4 R2. * 4
        \end{minted}

        \begin{comment}
        <lilypond>
        {
            R1 | R1 * 3 | \time 3/4 R2. * 4
        }
        </lilypond>
        \end{comment}

        \begin{abjadbookoutput}
        \noindent\includegraphics[max width=\linewidth, right,]{assets/lilypond-997b86f1b2e6999a00de261e7d942b05.pdf}
        \end{abjadbookoutput}

    \item[Typographic overrides:]
        \hfill \vspace{-\baselineskip} \\
        \begin{multicols}{2}

            \begin{minted}{text}
            c'8 [ d'
            \override NoteHead.style = #'cross e' f'
            \once \override NoteHead.style = #'harmonic g'
            a' \revert NoteHead.style b' c'' ]
            \end{minted}

            \begin{comment}
            <lilypond>[stylesheet=stylesheet-minimal.ily]
            \new Staff {
                c'8 [ d'
                \override NoteHead.style = #'cross e' f'
                \once \override NoteHead.style = #'harmonic g'
                a' \revert NoteHead.style b' c'' ]
            }
            </lilypond>
            \end{comment}

            \begin{abjadbookoutput}
            \noindent\includegraphics[max width=\linewidth, right,]{assets/lilypond-9cb24b4a763563e71705255aba542a41.pdf}
            \end{abjadbookoutput}

        \end{multicols}

\end{description}

\hrulefill{}

\section{Nested Contexts}

LilyPond can \enquote{nest} musical containers --- called
\enquote{contexts}\footnote{%
LilyPond provides a variety of different common
contexts out of the box which handle many compositional use-cases, and allows
users to define their own. Custom contexts can express their own typographic
styles, e.g.\ smaller musical fonts for cue voices, or drastic graphical
changes to create graphic notation. Contexts can also be \enquote{named}, which
supports LilyPond's context concatenation behavior. } --- together to create
examples of parallel music. The following example creates the structure
necessary to typeset a typical Bach chorale:

\begin{minted}{text}
\new Score <<
    \new ChoirStaff <<
        \new Staff = "Upper" <<
            \clef "treble"
            \new Voice = "Soprano" { \voiceOne c''2 d''2 }
            \new Voice = "Alto" { \voiceTwo c'2 d'2 }
        >>
        \new Staff = "Lower" <<
            \clef "bass"
            \new Voice = "Tenor" { \voiceOne c2 d2 }
            \new Voice = "Bass" { \voiceTwo c,2 d,2 }
        >>
    >>
>>
\end{minted}

\end{document}