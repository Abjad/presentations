\documentclass[a4paper]{tufte-handout}

\newcommand{\headerLeft}{\textsc{Abjad/CCRMA}}
\newcommand{\headerCenter}{\textsc{Bash Quick Reference}}
\newcommand{\headerRight}{\textsc{August $|$ 2017}}

\usepackage{csquotes}
\usepackage{multicol}
\usepackage{graphicx}
\usepackage{lab_notes}
\usepackage{fancyvrb}
\fvset{fontsize=\normalsize}
\usepackage{minted}
\usemintedstyle{bw}
\setminted{%
    autogobble=true,
    fontsize=\small,
}

\title{Bash Quick Reference}
\date{2017}

\begin{document}
\maketitle

\section{Path parts in a Unix-like system}

\begin{description}
    \item [\texttt{.}]
        \hfill The current directory
    \item [\texttt{..}]
        \hfill The parent directory
    \item [\texttt{/}]
        \footnote{On Windows: \texttt{\textbackslash}}
        \hfill The directory \enquote{separator}
    \item [\textasciitilde]
        \hfill Your \enquote{home} directory
    \item [\texttt{*}] \hfill A wildcard, matching part of a path \end{description}

\hrulefill{}

\section{Navigating directories}

\begin{description}
    \item [\texttt{pwd}]
        \footnote{On Windows: \texttt{chdir}}
        \hfill Print the current directory
    \item [\texttt{ls}]
        \footnote{On Windows: \texttt{dir} or \texttt{tree}}
        \hfill List information about files and directories
        \begin{minted}{bash}
        $ ls              # list the current directory's contents
        $ ls -l           # list with extra information
        $ ls -a           # list hidden files too
        $ ls *.txt        # list files ending with .txt
        $ ls ~/../*.txt   # list files ending with .txt in the parent of your home folder
        \end{minted}
    \item [\texttt{cd}]
        \footnote{Same on Windows}
        \hfill Change the current directory
        \begin{minted}{bash}
        $ cd foo          # change to the foo/ directory in the current directory
        $ cd /foo         # change to the root /foo directory
        $ cd ~            # change to your home directory
        $ cd              # (also) change to your home directory
        $ cd ..           # go up one directory level
        \end{minted}
\end{description}

\hrulefill{}

\section{Getting help}

Most command-line programs will print out help information when invoked without
arguments. To get help explicitly, or to get more extensive output, try the
following:

\begin{description}
    \item [\texttt{<command> -h}]
        \hfill Display a help message about a command
    \item [\texttt{<command> -{}-help}]
        \hfill
    \item [\texttt{<command> help}]
        \hfill
    \item [\texttt{man <command>}]
        \footnote{On Windows: \texttt{help}}
        \hfill Help manual
\end{description}

\hrulefill{}

\section{Printing and deleting}

\begin{description}
    \item [\texttt{echo}]
        \footnote{Same on Windows}
        \hfill Print a message
        \begin{minted}{bash}
        $ echo "Hello, world!"
        $ echo "Hello, world!" > hello.txt
        $ echo "Hello, again!" >> hello.txt
        \end{minted}
    \item [\texttt{clear}]
        \footnote{On Windows: \texttt{cls}}
        \hfill Clear the terminal screen
\end{description}

\hrulefill{}

\section{Making, moving, copying}

\begin{description}
    \item [\texttt{mkdir}]
        \footnote{Same on Windows}
        \hfill Create a directory
        \begin{minted}{bash}
        $ mkdir foo
        $ mkdir foo/bar
        $ mkdir -p one/two/three
        \end{minted}
    \item [\texttt{mv}]\footnote{
        The Unix \texttt{mv} command both renames files and changes their
        location. On Windows, \texttt{move} will move a file to a different
        directory, but \texttt{ren} will change it's name and or directory.}
        \hfill Rename a file or directory
        \begin{minted}{bash}
        $ mv old-name.txt new-name.txt
        $ mv source.txt target/
        \end{minted}
    \item [\texttt{cp}]
        \footnote{On Windows: \texttt{copy} or \texttt{xcopy}}
        \hfill Copy a file or directory
        \begin{minted}{bash}
        $ cp source.txt target.txt
        $ cp -R source/ target
        \end{minted}
    \item [\texttt{touch}]
        \hfill Create an empty file \\
        \hfill Or update the timestamp of an existing file
\end{description}

\hrulefill{}

\section{Editing and opening}

\begin{description}
    \item [\texttt{nano}]
        \footnote{Not available on Windows, but try \texttt{edit}}
        \hfill A minimal text editor
    \item [\texttt{vim}]
        \hfill A \enquote{programmer's} text editor
    \item [\texttt{open}]
        \footnote{On Windows: \texttt{start}.
        Some Linux systems use \texttt{xdg-open} instead.}
        \hfill Open a file in its default application
\end{description}

\hrulefill{}

\section{Deleting}

\begin{description}
    \item [\texttt{rmdir}]
        \footnote{Same on Windows}
        \hfill Remove folder(s)
    \item [\texttt{rm}]
        \footnote{On Windows: \texttt{del}}
        \footnote{Be careful! This command removes files and directories
        permanently. And combining the \texttt{-R} and \texttt{-f} flags can
        delete many files at once.}
        \hfill Remove file(s)
        \begin{minted}{bash}
        $ rm delete-me.txt
        $ rm -R delete-me/  # be careful!
        \end{minted}
\end{description}

\hrulefill{}

\section{Reading files}

\begin{description}
    \item [\texttt{cat}]
        \footnote{On Windows: \texttt{type}}
        \hfill Print file contents
    \item [\texttt{head}]
        \hfill Print the beginning of file(s)
    \item [\texttt{tail}]
        \hfill Print the end of file(s)
    \item[\texttt{less}]
        \footnote{On Windows: \texttt{more}}
        \hfill Display output one screen at a time
\end{description}

\hrulefill{}

\section{Searching and sorting}

\begin{description}
    \item [\texttt{which}]
        \hfill Search the user's \texttt{\$PATH} for a program file
        \begin{minted}{bash}
        $ which lilypond
        \end{minted}
    \item [\texttt{sort}]
        \footnote{Same on Windows}
        \hfill Sort lines
        \begin{minted}{bash}
        $ echo -e "one\ntwo\nthree" > lines.txt
        $ cat lines.txt
        $ cat lines.txt | sort
        \end{minted}
    \item [\texttt{grep}]
        \footnote{On Windows: \texttt{find}}
        \hfill Search files for lines that match a pattern
        \begin{minted}{bash}
        $ echo -e "one\ntwo\nthree" > lines.txt
        $ cat lines.txt | grep t
        \end{minted}
    \item [\texttt{ack}]
        \hfill Like \texttt{grep}, but nicer
    \item [\texttt{find}]
        \footnote{The single quotes around \texttt{*{}.txt} prevent Bash from
        prematurely expanding the \emph{pattern} into multiple file names.}
        \hfill Search for files that meet a desired criteria
        \begin{minted}{bash}
        $ find . -name '*.txt'
        \end{minted}
\end{description}

\hrulefill{}

\section{Permissions}

\begin{description}
    \item [\texttt{sudo}]
        \hfill Execute a command as another user
    \item [\texttt{chown}]
        \hfill Change file owner and group
    \item [\texttt{chmod}]
        \hfill Change access permissions
\end{description}

\hrulefill{}

\section{Glue}

\begin{description}
    \item [\texttt{CMD1 | CMD2}]
        \hfill Send output from one command to another
    \item [\texttt{CMD > file}]
        \hfill Write the output of a command to a file
    \item [\texttt{CMD >{}> file}]
        \hfill \emph{Append} the output of a command to a file
    \item [\texttt{\$VARIABLE}]
        \hfill Bash variables start with \texttt{\$}
    \item [\texttt{'...'} vs \texttt{"..."}]
        \hfill Bash treats single and double quotes different!
\end{description}

\end{document}
