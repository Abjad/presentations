% $Header: /Users/joseph/Documents/LaTeX/beamer/solutions/conference-talks/conference-ornate-20min.en.tex,v 90e850259b8b 2007/01/28 20:48:30 tantau $

\documentclass{beamer}
\usepackage{hyperref}
\usefonttheme{serif} % default family is serif
%\usepackage{fontspec}
%\setmainfont{Times New Roman}
% This file is a solution template for:

% - Talk at a conference/colloquium.
% - Talk length is about 20min.
% - Style is ornate.

% Copyright 2004 by Till Tantau <tantau@users.sourceforge.net>.
%
% In principle, this file can be redistributed and/or modified under
% the terms of the GNU Public License, version 2.
%
% However, this file is supposed to be a template to be modified
% for your own needs. For this reason, if you use this file as a
% template and not specifically distribute it as part of a another
% package/program, I grant the extra permission to freely copy and
% modify this file as you see fit and even to delete this copyright
% notice. 

\usepackage{pdfpages}
\mode<presentation>
{
  \usetheme{Warsaw}
  % or ...
  \setbeamercovered{transparent}
  % or whatever (possibly just delete it)
}
\usepackage[english]{babel}
% or whatever


\usepackage{times}
\usepackage[T1]{fontenc}
% Or whatever. Note that the encoding and the font should match. If T1
% does not look nice, try deleting the line with the fontenc.


\title[] % (optional, use only with long paper titles)
{Abjad: an open-source software system for formalized score control}

\subtitle
{Introductory Workshop}

\author[]{Trevor Ba\v{c}a \inst{1} \and Josiah Oberholtzer \inst{1} \and Jeffrey Trevi\~{n}o \inst{2}}
\institute[shortinst]{\inst{1}Department of Music \\ Harvard University \and \inst{2} Department of Music \\ Colorado College}

%\author[shortname]{author1 \inst{1} \and author2 \inst{2}}
%\institute[shortinst]{\inst{1} affiliation for author1 \and %
             %         \inst{2} affiliation for author2}
% - Give the names in the same order as the appear in the paper.
% - Use the \inst{?} command only if the authors have different
%   affiliation.

%\institute[University of California, San Diego] % (optional, but mostly needed)
%{
  %\inst{1}%
 % Department of Music\\
%  University of Californi , San Diego}
% - Use the \inst command only if there are several affiliations.
% - Keep it simple, no one is interested in your street address.

\date[CFP 2003] % (optional, should be abbreviation of conference name)
{Study Day on Computer Simulation of Musical Creativity (Saturday 27 June 2015)}
% - Either use conference name or its abbreviation.
% - Not really informative to the audience, more for people (including
%   yourself) who are reading the slides online

\subject{Music}
% This is only inserted into the PDF information catalog. Can be left
% out. 



% If you have a file called "university-logo-filename.xxx", where xxx
% is a graphic format that can be processed by latex or pdflatex,
% resp., then you can add a logo as follows:

 \pgfdeclareimage[height=1.5cm,width=1.5cm]{university-logo}{university-logo}
 \logo{\pgfuseimage{university-logo}}



% Delete this, if you do not want the table of contents to pop up at
% the beginning of each subsection:
%\AtBeginSubsection[]
%{
  %\begin{frame}<beamer>{Outline}
    %\tableofcontents[currentsection,currentsubsection]
  %\end{frame}
%}


% If you wish to uncover everything in a step-wise fashion, uncomment
% the following command: 

%\beamerdefaultoverlayspecification{<+->}


\begin{document}

\begin{frame}
  \titlepage
\end{frame}

\begin{frame}{Introduction}
  \tableofcontents
  % You might wish to add the option [pausesections]
\end{frame}


% Structuring a talk is a difficult task and the following structure
% may not be suitable. Here are some rules that apply for this
% solution: 

% - Exactly two or three sections (other than the summary).
% - At *most* three subsections per section.
% - Talk about 30s to 2min per frame. So there should be between about
%   15 and 30 frames, all told.

% - A conference audience is likely to know very little of what you
%   are going to talk about. So *simplify*!
% - In a 20min talk, getting the main ideas across is hard
%   enough. Leave out details, even if it means being less precise than
%   you think necessary.
% - If you omit details that are vital to the proof/implementation,
%   just say so once. Everybody will be happy with that.

\section{Rhythm Makers}
\section{Abjad?}
\section{Score Applications}
\section{Non-score Applications}

\begin{frame}{Abjad?}
  \begin{itemize}
  \item
    C into Finale via MIDI (1997)
    \item
    Mathematica into Sibelius via MIDI (2001)
     \item
    Mathematica into SCORE (2003)
     \item
    Mathematica into LilyPond (2004)
     \item
    Python into Adobe Illustrator (2004)
     \item
    Python into LilyPond (2005)
     \item
    Public release on GoogleCode (2008)
      \item
    Migration to GitHub (2011)
  \end{itemize}
\end{frame}

\begin{frame}{Contacts}
  \begin{itemize}
  \item
    Individual Contacts
    \begin{itemize}
    \item
    Trevor Ba\v{c}a -- www.trevorbaca.com -- trevor.baca@gmail.com -- www.github.com/trevorbaca
    \item
    Josiah Oberholtzer -- www.josiahwolfoberholtzer.com -- josiah.oberholtzer@gmail.com --github.com/josiah-wolf-oberholtzer
      \item
  Jeffrey Trevi\~{n}o -- www.jeffreytrevino.com -- jeffrey.trevino@gmail.com -- https://github.com/jefftrevino
  \end{itemize}
  \end{itemize}
\end{frame}


\end{document}


